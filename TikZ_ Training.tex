\documentclass[book,openany]{jlreq}

%---------- package ----------% 
\usepackage{graphicx}
\usepackage[pdfencoding=auto]{hyperref}
\usepackage{booktabs}
%\usepackage{subfig}
\usepackage{pifont}
\usepackage{url}
\usepackage{cite}
\usepackage{ulem}
\usepackage{siunitx}
\usepackage{float}
\usepackage{tcolorbox}
\tcbuselibrary{breakable}
\usepackage{cancel}
\usepackage{color}

%--- TikZ ---%
\usepackage{tikz}
\usepackage{circuitikz}
\usetikzlibrary{calc}

%--- physics2 ---%
\usepackage{physics2}
\usephysicsmodule{ab}   % 括弧サイズの自動調整
\usephysicsmodule{ab.braket}    % braket記法
\usephysicsmodule{diagmat}  % 対角行列
\usephysicsmodule[showleft=2,showtop=2]{xmat}   % n×m行列
%--- physics2 END ---%
% 数式環境
\usepackage{amsmath,amssymb,amsthm}
% ベクトル
\usepackage{bm}
%--- 微分演算子 ---%
\usepackage{fixdif}
\usepackage{derivative}
%--- 微分演算子 END ---%
% 花文字
\usepackage{mathrsfs}
%
%---------- package END ----------% 

%---------- newcommand ----------% 
% 積分値の評価
\newcommand{\eval}[1]{\left.#1\right|}
% Landau記号
\newcommand{\order}[1]{\mathcal{O}\ab(#1)}
%--- physicsパッケージの\vbコマンドを再現 --%
\makeatletter
\newcommand\vb{\@ifstar\boldsymbol\mathbf}
\newcommand\va[1]{\@ifstar{\vec{#1}}{\vec{\mathrm{#1}}}}
\newcommand\vu[1]{%
\@ifstar{\hat{\boldsymbol{#1}}}{\hat{\mathbf{#1}}}}
\makeatother
%--- physicsパッケージの\vbコマンドを再現 END --%
%--- 勾配・発散・回転 ---%
% 勾配
\DeclareMathOperator{\grad}{\nabla}
% 発散
% \divが「÷」と競合するため再定義
\DeclareMathOperator{\divergence}{\nabla\cdot}
\let\divisionsymbol\div 
\renewcommand{\div}{\divergence} 
% 回転
\DeclareMathOperator{\rot}{\nabla\times}
%--- 勾配・発散・回転 END ---%
% 実部
\renewcommand{\Re}{\operatorname{Re}}
% 虚部
\renewcommand{\Im}{\operatorname{Im}}
% トレース
\newcommand{\Tr}{\operatorname{Tr}}
% rank
\newcommand{\rank}{\operatorname{rank}}
% \mqtyコマンド
\newcommand{\mqty}[1]{\begin{matrix}#1\end{matrix}}
%
\renewcommand{\CancelColor}{\color{red}}
%
%---------- newcommand END ----------% 

% \usepackage[hang,small,bf]{caption}
% \usepackage[subrefformat=parens]{subcaption}
% \captionsetup{compatibility=false}

\theoremstyle{definition}
\newtheorem{theorem}{定理}[chapter]
\newtheorem*{theorem*}{定理}
\newtheorem{definition}[theorem]{定義}
\newtheorem*{definition*}{定義}

\usepackage{xcolor}
\hypersetup{
  bookmarksnumbered=true,
  colorlinks=true,
  citecolor=red,
  linkcolor=blue,
  urlcolor=orange,
}

\begin{document}
\title{TikZ練習帳}
\author{吉本 伸一}
\maketitle
\tableofcontents

\chapter{Lesson 1}
\section{直線の描画}

\begin{tikzpicture}
    \draw (1,-1)--(3,-1.5);
\end{tikzpicture}

閉じた直線(三角形):\par
\tikz \draw (0,0)--(2,1)--(3,-1)--cycle;

垂線: \par
\begin{tikzpicture}
    \draw (0,0)--(3,-1);
    \draw (1,1)--($(0,0)!(1,1)!(3,-1)$);
\end{tikzpicture}

長方形: \par
\tikz \draw (0,0) rectangle (3,1);

円: \par
\tikz \draw(0,0) circle (1);

楕円:\par
\tikz \draw (0,0) circle [x radius=2, y radius=1, rotate=30];

円弧(座標) (偏角1,偏角2,半径):\par
\tikz \draw (0,0) arc (30:90:2);

円弧 (座標1) to [偏角1,偏角2] (座標2):\par
\tikz \draw (0,0) to [out=60,in=120] (3,0);

円弧 の回転:\par
\tikz \draw[rotate=45] (0,0) to [out=45,in=135] (2,0);

任意の曲線  (座標1) to [偏角1,偏角2] (座標2):\par
\tikz \draw (0,0) to [out=45,in=135] (2,2);

放物線:\par
\tikz \draw (-2,-2) parabola bend (0,0) (2,-2);

\section{Figure環境での使用}

\begin{figure}[htbp]
    \centering

    \begin{tikzpicture}

        \tikz \draw (0,0)--(2,1)--(3,-1)--cycle;

    \end{tikzpicture}

    \caption{三角形}
\end{figure}

\begin{figure}[htbp]
    \centering
    \begin{tabular}{cc}

        \begin{minipage}[t]{0.3\linewidth}

            \begin{tikzpicture}
                \draw (0,0)--(3,0);
                \draw (1,1)--($(0,0)!(1,1)!(3,0)$);
            \end{tikzpicture}
            \caption{垂線}
        \end{minipage} &

        \begin{minipage}[t]{0.3\linewidth}
            \begin{tikzpicture}
                \draw (0,0) circle [x radius=2, y radius=1, rotate=30];
            \end{tikzpicture}
            \caption{円}
        \end{minipage}
    \end{tabular}
\end{figure}

\section{線のスタイル}

直接指定 \par
\tikz \draw[line width=1pt] (0,0)--(2,0);

ultra thin:0.1pt \par
\tikz \draw[ultra thin] (0,0)--(2,0);

very thin:0.2pt \par
\tikz \draw[very thin] (0,0)--(2,0);

thin:0.4pt \par
\tikz \draw[thin] (0,0)--(2,0);

semithick:0.6pt \par
\tikz \draw[semithick] (0,0)--(2,0);

thick:0.8pt \par
\tikz \draw[thick] (0,0)--(2,0);

very thick:1.0pt \par
\tikz \draw[very thick] (0,0)--(2,0);

ultra thick:1.2pt \par
\tikz \draw[ultra thick] (0,0)--(2,0);

\subsection{線種}

dotted \par
\tikz \draw[dotted] (0,0)--(2,0);

densely dotted \par
\tikz \draw[densely dotted] (0,0)--(2,0);

loosely dotted \par
\tikz \draw[loosely dotted] (0,0)--(2,0);

dashed \par
\tikz \draw[dashed] (0,0)--(2,0);

densely dashed \par
\tikz \draw[densely dashed] (0,0)--(2,0);

loosely dashed \par
\tikz \draw[loosely dashed] (0,0)--(2,0);

dash dot \par
\tikz \draw[dash dot] (0,0)--(2,0);

densely dash dot \par
\tikz \draw[densely dash dot] (0,0)--(2,0);

loosely dash dot \par
\tikz \draw[loosely dash dot] (0,0)--(2,0);

dash dot dot \par
\tikz \draw[dash dot dot] (0,0)--(2,0);

densely dash dot dot \par
\tikz \draw[densely dash dot dot] (0,0)--(2,0);

loosely dash dot dot \par
\tikz \draw[loosely dash dot dot] (0,0)--(2,0);

直接パターン指定 \par
\tikz \draw[dash pattern=on 2pt off 3pt on 4pt off 4pt](0,0)--(3.5cm,0);

\subsection{線の修飾}

2重線: \par
\tikz \draw[double] (0,0)--(2,0);

2重線距離: \par
\tikz \draw[double distance=5pt] (0,0)--(2,0);

片側矢印: \par
\tikz \draw[->] (0,0)--(2,0);

片側矢印: \par
\tikz \draw[<-] (0,0)--(2,0);

両側矢印: \par
\tikz \draw[<->] (0,0)--(2,0);

2重矢印: \par
\tikz \draw[->>] (0,0)--(2,0);

2重線矢印: \par
\tikz \draw[double, <->] (0,0)--(2,0);

cap:round \par
\tikz \draw[line width=5pt, line cap=round] (0,0)--(2,0);

cap:butt \par
\tikz \draw[line width=5pt, line cap=butt] (0,0)--(2,0);

cap:rect \par
\tikz \draw[line width=5pt, line cap=rect] (0,0)--(2,0);

Red:\par
\tikz \draw[red] (0,0)--(2,0);

Blue:\par
\tikz \draw[blue] (0,0)--(2,0);

Green:\par
\tikz \draw[green] (0,0)--(2,0);

Cyan:\par
\tikz \draw[cyan] (0,0)--(2,0);

Magenta:\par
\tikz \draw[magenta] (0,0)--(2,0);

Yellow:\par
\tikz \draw[yellow] (0,0)--(2,0);

White:\par
\tikz \draw[white] (0,0)--(2,0);

Black:\par
\tikz \draw[black] (0,0)--(2,0);

Gray:\par
\tikz \draw[gray] (0,0)--(2,0);

Lightgray:\par
\tikz \draw[lightgray] (0,0)--(2,0);

Darkgray:\par
\tikz \draw[darkgray] (0,0)--(2,0);

Olive:\par
\tikz \draw[olive] (0,0)--(2,0);

Lime:\par
\tikz \draw[lime] (0,0)--(2,0);

Brown:\par
\tikz \draw[brown] (0,0)--(2,0);

Orange:\par
\tikz \draw[orange] (0,0)--(2,0);

Purple:\par
\tikz \draw[purple] (0,0)--(2,0);

Violet:\par
\tikz \draw[violet] (0,0)--(2,0);

Pink:\par
\tikz \draw[pink] (0,0)--(2,0);

Teal:\par
\tikz \draw[teal] (0,0)--(2,0);

% Bittersweet(定義):\par
%\tikz \draw[bittersweet] (0,0)--(2,0);

\begin{tikzpicture}
    \draw (-1.5,0) --(1.5,0); \draw (0,-1.5) --(0,1.5);
    \draw (0,0) circle (1cm);
\end{tikzpicture}

\begin{tikzpicture}
    \draw (-1.5,0) --(1.5,0); \draw (0,-1.5) --(0,1.5);
    \draw (0,0) circle (1cm);
    \draw[step=.5cm,gray,very thin] (-1.4,-1.4) grid (1.4,1.4);
\end{tikzpicture}

\begin{tikzpicture}
    \draw (0,0) circle (1cm);
    \filldraw[fill=green!20!white,draw=green!50!black](0,0)--(3mm,0mm)arc(0:30:3mm)--cycle;
    \draw[red,very thick](30:1cm)-- node[left=1pt]{$\sin\alpha$} +(0,-0.5);
    \draw[blue,very thick](30:1cm)++(0,-0.5) -- node[below=2pt]{$\cos\alpha$} (0,0);
    \draw (30:1cm) -- (0,0);
\end{tikzpicture}

\begin{tikzpicture}[scale=2]
    %\draw[-Stealth] (-1.5,0) --(1.5,0); \draw[-Stealth] (0,-1.5) --(0,1.5);
    \draw (0,0) circle (1cm);
    \draw[step=.5cm,gray,very thin] (-1.4,-1.4) grid (1.4,1.4);
    \filldraw[fill=green!20!white,draw=green!50!black](0,0)--(3mm,0mm)arc(0:30:3mm)--cycle;
    \draw[red,very thick](30:1cm)-- node[left=1pt] {$\sin\alpha$} +(0,-0.5);
    \draw[blue,very thick](30:1cm)++(0,-0.5) -- node[below=2pt]{$\cos\alpha$} (0,0);
    \draw (30:1cm) -- (0,0);
\end{tikzpicture}

\chapter{電気回路の作図}
\section{CircuTikz}

\begin{figure}[htbp]
    \begin{center}
        \begin{circuitikz}[american currents]
            \draw (0,0)
            to[sV=$V$] (0,2)
            to[short] (2,2)
            to[european resistor=$R$] (2,0)
            to[short] (0,0);
            \draw (2,2)
            to[short] (4,2)
            to[L=$L$] (4,0)
            to[short] (2,0);
            \draw (4,2)
            to[short] (6,2)
            to[C=$C$] (6,0)
            to[short] (4,0);
        \end{circuitikz}
        \caption{RLC並列回路}
    \end{center}
\end{figure}

\begin{figure}
    \begin{center}
        \begin{circuitikz}
            \draw(0,0) to[R=$R_1$,i=$i_1$,v=$v_1$] (2,0);
        \end{circuitikz}
        \caption{抵抗(american)}
    \end{center}
\end{figure}

\begin{figure}
    \begin{center}
        \begin{circuitikz}
            \draw(0,0) to[european resistor=$R_1$] (2,0);
        \end{circuitikz}
        \caption{抵抗(european)}
    \end{center}
\end{figure}


\tikz \draw(0,0) to[american inductor=$L_1$] (2,0);


\tikz \draw(0,0) to[cute inductor=$L_1$] (2,0);

\tikz \draw(0,0) to[C=$C_1$] (2,0);

\tikz \draw(0,0) to[sV=$V_1$] (2,0);

\tikz \draw(0,0) to[battery=5V] (2,0);

\section{可変抵抗素子}

\begin{figure}[htbp]
    \begin{center}
        \begin{circuitikz}
            \draw(0,0) to[vR] (2,0);
            \draw(3,0) to[pR] (5,0);
            \draw(6,0) to[sR] (8,0);
        \end{circuitikz}
        \caption{可変抵抗(american)}
    \end{center}
\end{figure}

\begin{figure}[htbp]
    \begin{center}
        \begin{circuitikz}[european]
            \draw(0,0) to[vR] (2,0);
            \draw(3,0) to[pR] (5,0);
            \draw(6,0) to[sR] (8,0);
        \end{circuitikz}
        \caption{可変抵抗(european)}
    \end{center}
\end{figure}

\begin{figure}[htbp]
    \begin{center}
        \begin{circuitikz}
            \draw (0,0) to[pR, name=A] (2,0);
            \draw (3,0) to[pR, wiper pos=0.8, name=B] (5,0);
        \end{circuitikz}
        \caption{可変抵抗(矢印位置)}
    \end{center}
\end{figure}

\begin{figure}[htbp]
    \begin{center}
        \begin{circuitikz}
            \ctikzset{resistors/zigs=9}
            \draw (0,0) to[pR, name=A] (2,0);
            \draw (3,0) to[pR, wiper pos=0.8, name=B] (5,0);
            \ctikzset{resistors/width=1.5,resistors/zigs=9}
            \draw (0,-1) to[pR, name=A] (4,-1);
            \draw (5,-1) to[pR, wiper pos=0.8, name=B] (9,-1);
        \end{circuitikz}
        \caption{可変抵抗(長)}
    \end{center}
\end{figure}

\begin{figure}[htbp]
    \begin{center}
        \begin{circuitikz}
            \draw (0,0) to[pR, wiper pos =0.2, name=P] ++(0,2);
            \draw (P.wiper) to[L] ++(-2,0);

            \draw (3,0) to[pR, wiper pos =0.2, name=P2] ++(0,2);
            \draw (P2.wiper) to[L,mirror] ++(-2,0);
        \end{circuitikz}
        \caption{可変抵抗と回路素子}
    \end{center}
\end{figure}

\begin{figure}[htbp]
    \begin{center}
        \begin{circuitikz}
            \draw (0,0) to[pR, wiper pos =0.2, name=P, mirror] ++(0,2);
            \draw (P.wiper) to[L] ++(2,0);
        \end{circuitikz}
        \caption{可変抵抗(反転)と回路素子}
    \end{center}
\end{figure}

\section{キャパシタ}

\begin{figure}[htbp]
    \begin{center}
        \begin{circuitikz}
            \draw(0,0) to[C] (2,0);
            \draw(3,0) to[cC] (5,0);
        \end{circuitikz}
        \caption{キャパシタ}
    \end{center}
\end{figure}

\begin{figure}[htbp]
    \begin{center}
        \begin{circuitikz}
            \draw(0,0) to[vC] (2,0);
            \draw(3,0) to[PZ] (5,0);
            \draw(6,0) to[sC] (8,0);
        \end{circuitikz}
        \caption{可変キャパシタ,ピエゾ,キャパシタセンサ}
    \end{center}
\end{figure}

\begin{figure}[htbp]
    \begin{center}
        \begin{circuitikz}
            \draw(0,0) to[eC] (2,0);
            \draw(5,0) to[eC,mirror] (3,0);
        \end{circuitikz}
        \caption{極性キャパシタ}
    \end{center}
\end{figure}

\section{インダクタ}

\begin{figure}[htbp]
    \begin{center}
        \begin{circuitikz}
            \draw(0,0) to[L] (2,0);
            \draw(3,0) to[vL] (5,0);
            \draw(6,0) to[sL] (8,0);
        \end{circuitikz}
        \caption{インダクタ}
    \end{center}
\end{figure}

\begin{figure}[htbp]
    \begin{center}
        \begin{circuitikz}[american]
            \draw(0,0) to[L] (2,0);
            \draw(3,0) to[vL] (5,0);
            \draw(6,0) to[sL] (8,0);
        \end{circuitikz}
        \caption{インダクタ(American)}
    \end{center}
\end{figure}

\begin{figure}[htbp]
    \begin{center}
        \begin{circuitikz}[european]
            \draw(0,0) to[L] (2,0);
            \draw(3,0) to[vL] (5,0);
            \draw(6,0) to[sL] (8,0);
        \end{circuitikz}
        \caption{インダクタ(European)}
    \end{center}
\end{figure}

\section{ダイオード}

\begin{figure}[htbp]
    \begin{center}
        \begin{circuitikz}
            \draw(0,0) to[D, l=Empty] (2,0);
            \draw(3,0) to[D*,l=Full] (5,0);
            \draw(6,0) to[D-,l=Stroke] (8,0);
        \end{circuitikz}
        \caption{ダイオード}
    \end{center}
\end{figure}

\begin{figure}[htbp]
    \begin{center}
        \begin{circuitikz}
            \draw(0,0) to[sD*,l=Schotkky] (2,0);
            \draw(3,0) to[zD*,l=Zener] (5,0);
            \draw(6,0) to[zzD*, l=ZZener] (8,0);
            \draw(9,0) to[tD*,l=Tunnel] (10,0);
        \end{circuitikz}
        \caption{各種ダイオード}
    \end{center}
\end{figure}

\begin{figure}[htbp]
    \begin{center}
        \begin{circuitikz}
            \draw(-1.5,0) to ++(1.5,0) to[D*] (1.5,-1.5);
            \draw(0,0) to[D*] (1.5,1.5);
            \draw(1.5,-1.5) to[D*] (3,0);
            \draw(1.5,1.5) to[D*] (3,0) to ++(1.5,0);

            \draw(-1.5,-1.7) to (1.5,-1.7);
            \draw(1.5,-1.7) to ++(0,0.2);
            \draw(-1.5,1.7) to (1.5,1.7);
            \draw(1.5,1.7) to ++(0,-0.2);
        \end{circuitikz}
        \caption{全波整流ダイオード}
    \end{center}
\end{figure}

\section{グラウンド}

\begin{figure}[htbp]
    \begin{center}
        \begin{circuitikz}
            \draw(0,0) to node[ground]{} ++(0,0);
            \draw(1,0) to node[tlground]{} ++(0,0);
            \draw(2,0) to node[eground]{} ++(0,0);
            \draw(3,0) to node[eground2]{} ++(0,0);
        \end{circuitikz}
        \caption{グラウンド}
    \end{center}
\end{figure}

\begin{figure}[htbp]
    \begin{center}
        \begin{circuitikz}
            \draw(0,0) to node[batteryshape]{} ++(1,0);
            \draw(2,0) to node[battery1shape]{}++(1,0);
            \draw(4,0) to node[battery2shape]{}++(1,0);
        \end{circuitikz}
        \caption{電池}
    \end{center}
\end{figure}

\begin{figure}[htbp]
    \begin{center}
        \begin{circuitikz}
            \draw(0,0) to [sqV] ++(1.5,0);
            \draw(2,0) to [tV] ++(1.5,0);
            \draw(4,0) to [pvsource] ++(1.5,0);
        \end{circuitikz}
        \caption{その他}
    \end{center}
\end{figure}

\begin{figure}[htbp]
    \begin{center}
        \begin{circuitikz}
            \draw(0,0) to [smeter] ++(1,0);
            \draw(2,0) to [qiprobe] ++(1,0);
            \draw(4,0) to [qvprobe] ++(1,0);
            \draw(6,0) to [qpprobe] ++(1,0);
        \end{circuitikz}
        \caption{メーター(ピクチャー)}
    \end{center}
\end{figure}

\begin{figure}[htbp]
    \begin{center}
        \begin{circuitikz}
            \draw(0,0) to [oscope] ++(1,0);
            \draw(2,0.5) to [iloop] ++(0,-1);
        \end{circuitikz}
        \caption{オシロスコープ・カレントループ}
    \end{center}
\end{figure}

\begin{figure}[htbp]
    \begin{center}
        \begin{circuitikz}[american]
            \draw (3,0) to [iloop, name=I] ++(0,-2) node[ground] (GND){};
            \draw (I.i) -- ++(0.5,0) node[oscopeshape, anchor=left, name=O]{};
            \draw (O.south) -- (O.south |- GND) node[ground]{};
        \end{circuitikz}
        \caption{オシロスコープ接続}
    \end{center}
\end{figure}

\begin{thebibliography}{9}
    \bibitem{Ono}
    M. Ono et al., TRISTAN RF system with normal conducting cavity, KEK Internal 87-6 (1987)
\end{thebibliography}
%
\end{document}